\presentationsection{\LaTeX template}

\begin{sectionslide}
  Starting a new section
\end{sectionslide}

%%%%%%%%%%%%%%%%%%%%%%%%%%%%%%%%%%%%%%%%%%%%%%%%%%%%%%%%%%%%%%%%%%%%%%%%%%%%%%%%
\presentationsubsection{Lists}

\begin{subsectionslide}
  Itemize

  \begin{itemize}
    \item First entry.
    \item Second entry.
    \item Third entry.
  \end{itemize}

  Enumerate:

  \begin{enumerate}
    \item First entry.
    \item Second entry.
    \item Third entry.
  \end{enumerate}
\end{subsectionslide}

%%%%%%%%%%%%%%%%%%%%%%%%%%%%%%%%%%%%%%%%%%%%%%%%%%%%%%%%%%%%%%%%%%%%%%%%%%%%%%%%
\presentationsubsection*{This sub section is not in the TOC?}

\begin{subsectionslide}
  Both \texttt{presentationsection} and \texttt{presentationsubsection} can
  be \textit{``stared''} so they do not appear in the table of content.
\end{subsectionslide}

%%%%%%%%%%%%%%%%%%%%%%%%%%%%%%%%%%%%%%%%%%%%%%%%%%%%%%%%%%%%%%%%%%%%%%%%%%%%%%%%
\presentationsubsection{Theorem}

\begin{subsectionslide}
  \begin{theorem}[Example]
    Here is how theorems are displayed.
  \end{theorem}
\end{subsectionslide}

%%%%%%%%%%%%%%%%%%%%%%%%%%%%%%%%%%%%%%%%%%%%%%%%%%%%%%%%%%%%%%%%%%%%%%%%%%%%%%%%
\presentationsubsection{Images}

\begin{subsectionslide}
  \begin{minipage}[h!]{0.45\textwidth}
    \begin{itemize}
      \item NIST:
      \begin{itemize}
        \item Gaithersburg campus.
      \end{itemize}
    \end{itemize}
  \end{minipage}\hfill
  \begin{minipage}[h!]{0.45\textwidth}
    \includegraphics[scale=0.2]{./img/example-image.png}
  \end{minipage}
\end{subsectionslide}

\begin{plainslide}
  \begin{figure}[H]
    \begin{center}
      \includegraphics[scale=0.4]{./img/example-image.png}
      \caption{Big images can be displayed on a plain slide}
    \end{center}
  \end{figure}
\end{plainslide}

%%%%%%%%%%%%%%%%%%%%%%%%%%%%%%%%%%%%%%%%%%%%%%%%%%%%%%%%%%%%%%%%%%%%%%%%%%%%%%%%
\presentationsubsection{Code}

\begin{subsectionslide}[containsverbatim]
  Displaying code requires pygments !

  \begin{listing}[H]
    \begin{minted}[frame=lines,framesep=2mm,baselinestretch=0.8,fontsize=\small,linenos]{C++}
      #include <iostream>

      int main(int, char **) {
        std::cout << "Hello, World!" << std::endl;
        return 0;
      }
    \end{minted}
    \caption{Hello world in C}
  \end{listing}

\end{subsectionslide}

%%%%%%%%%%%%%%%%%%%%%%%%%%%%%%%%%%%%%%%%%%%%%%%%%%%%%%%%%%%%%%%%%%%%%%%%%%%%%%%%
\presentationsubsection{Table}

\begin{subsectionslide}

  \begin{center}
    \begin{table}
      \begin{tabular}{|c|c|c|c|}
       \hline
       \rowcolor{tableFirstRowColor} Col1 & Col2 & Col2 & Col3 \\ [0.5ex]
       \hline
       \cellcolor{tableFirstColColor} 1 & 6 & 87837 & 787 \\
       \hline
       \cellcolor{tableFirstColColor} 2 & 7 & 78 & 5415 \\
       \hline
       \cellcolor{tableFirstColColor} 3 & 545 & 778 & 7507 \\
       \hline
      \end{tabular}
      \caption{Example table}
    \end{table}
  \end{center}

\end{subsectionslide}

%%%%%%%%%%%%%%%%%%%%%%%%%%%%%%%%%%%%%%%%%%%%%%%%%%%%%%%%%%%%%%%%%%%%%%%%%%%%%%%%
\presentationsubsection{References}

\begin{subsectionslide}
  \begin{itemize}
    \item Paper 1 \cite{blattner2017model}
    \item Paper 2 \cite{bardakoff2020hedgehog}
  \end{itemize}
\end{subsectionslide}
